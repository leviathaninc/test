\documentclass[letterpaper,10pt]{article}
\usepackage{amsmath,amssymb,amsxtra,amsthm,microtype}
\usepackage{marvosym}
\usepackage{pgfplots}
\usepackage{mathtools}
\usepackage{graphicx}
\usepackage{tikz}
\usepackage{bbold}
\usepackage{xcolor}

%\color{lightgray}
%\pagecolor{darkgray}

\theoremstyle{plain}
\newtheorem{theorem}{Theorem}
\newtheorem{corollary}{Corollary}
\newtheorem{proposition}{Proposition}

\theoremstyle{definition}
\newtheorem{definition}{Definition}
\newtheorem{example}{Example}
\newtheorem{remark}{Remark}

\pgfplotsset{width=5cm, height=5cm, compat=1.9}

\usepackage[margin=2.0cm]{geometry}

\begin{document}

  \noindent\textbf{{\large Modern Alegbra \\Homework 6}}
  \smallskip
  
  \noindent\textbf{Name: }Parker Lockary\\
  
  \begin{enumerate}
    \item[5.4]
      \begin{enumerate}
      \item[(12)] Let $\sigma \in S_n$ have order $n$. Show that for all integers $i$ and $j$, $\sigma^i = \sigma^j$ if and only if $i \equiv j \mod n$.
      \begin{proof}
        \item[$\Rightarrow$]
        Suppose that $\sigma^i=\sigma^j$. Then $\sigma^i=\sigma^j\sigma^n,$ since $\sigma^n=\text{id},$ so $\sigma^n=\sigma^kn$ for $k\in\mathbb{N}_0$, so $\sigma^i=\sigma^j\sigma^{kn}$. Thus $\sigma^j\sigma^{kn}=\sigma^{j+kn}$. This implies that $i=j+kn,$ so $i\equiv j \mod n$.
        \item[$\Leftarrow$]
        Next, suppose that $i\equiv j \mod n,$ so $i=j+kn$ for $k \in \mathbb{N}_0$. If $k=0,$ then $i=j,$ so $\sigma^i = \sigma^j$. If $k>0$, then $\sigma^i=\sigma^{j+kn}=\sigma^j\sigma^{kn}$ for $k \in \textbb{N}$. But because $\sigma$ has order $n$,  $\sigma^{kn}=\sigma^{n}\sigma^{n(k-1)}=\text{id} \sigma^{n(k-1)}\sigma^{n(k-2)}=\cdots=\text{id}\sigma^{1n}=\text{id}^2=\text{id}$, so $\sigma^i=\sigma^{j+kn}=\sigma^j\sigma^{kn}=\text{id}\sigma^j=\sigma^j$ and therefore $\sigma^i=\sigma^j$.
      \end{proof}
      \item[(13)] Let $\sigma=\sigma_1 \cdots \sigma_m \in S_n$ be the product of disjoint cycles. Prove that the order of $\sigma$ is the least common multiple of the lengths of the cycles $\sigma_1, \cdots , \sigma_m$.
      \begin{proof}
        Let the order of $\sigma$ be $n$. Since the cycles are disjoint, they commute and so $(\sigma_1\cdots\sigma_m)^n=\sigma_1^n\cdots\sigma_m^n$. Let the length of each $\sigma_i^n$ be $k_i$. $\sigma_i$ is the identity permutation if and only if $k_i|n$. Therefore, the order of $\sigma$ is the smallest integer $n$ that is divisible be the lengths of all the cycles $\sigma_1, \cdots, \sigma_m$. That is the definition of the least common multiple, and so the order of $\sigma$ is the least common multiple.
      \end{proof}
    \end{enumerate}
  \end{enumerate}
\end{document}\
